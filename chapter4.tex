\chapter{Your Chapter Title Goes Here }

Distributed Acoustic Sensing (DAS) is a transformative technology in geosciences and engineering. DAS records ground motion along fiber-optic cables that are comparable to those obtained by single-component accelerometers or geophones. The transformative potential arises from the fiber itself being the sensor and allowing for a spatially continuous measurement. The fiber can be tens of kilometers in length and it can be located in shallowly buried trenches, in boreholes, or in some combination. The fiber geometry can encompass a large volume that can be tens of cubic kilometers in size. DAS inherently possesses properties of a large-N seismic array. The rapidly increasing interest in DAS arises from its potential to be used in continuous arrays that are kilometers in length while providing spatial resolution of meters and frequency response from millihertz to kilohertz.

DAS applications in geosciences and engineering are numerous and growing including opportunities for deploying early warning systems for earthquakes, volcanic eruptions, continental and marine landslides, and avalanches, and for monitoring reservoirs and civil infrastructure. DAS can complement and supplement conventional seismic sensors and arrays already used across a wide range of disciplines.

\section{Statistical Analysis}

YOUR EQUATIONS GOES GOOD

The error between predicted and actual pore pressure has been calculated as \( R^2 \) coefficient. The Coefficient of Determination  \( R^2 \), indicates the goodness of fit of the regression line to the actual data points.

 \( R^2 \) ranges from 0 to 1, where:
 

1. Sum of Squares Total (SST):

   \[ SST = \sum_{i=1}^{n} (y_i - \bar{y})^2 \]

2. Sum of Squares Regression (SSR):

   \[ SSR = \sum_{i=1}^{n} (\hat{y}_i - \bar{y})^2 \]

\newpage
